% See LICENSE file for copyright and license details.

\documentclass{resume}

\hypersetup
{
	pdfauthor={Marc-Antoine Loignon},
	pdftitle={Curriculum Vitae},
	pdfkeywords={CV, Curriculum Vitae, Développer Web}
}

\begin{document}
	\begin{multicols}{2}
		\section*{Marc-Antoine Loignon}
		\job{Développeur Web \& Pigiste}

		\columnbreak

		\begin{flushright}
			\scriptsize
			\begin{tabular}{cc}
				\address{Québec, Québec, Canada}
				\phone{(418) 808-XXXX}
				\email{developer@lognoz.org}
			\end{tabular}
		\end{flushright}
	\end{multicols}

	\vspace{2.5mm}

	\begin{multicols}{2}
		\subsection*{Projets professionels}
			\subsubsection*{Juin 2017 – Présent \\ Begin Edition International – Gestion de la relation client}
			\begin{itemize}
				\scriptsize
				\item Programmation PHP et Javascript du logiciel.
				\item Transformer une base de données filemaker en MySQL.
				\item Création d'un Entreprise Resource Planning (ERP).
				\item Communiquer avec les employés afin de comprendre leur besoins.
				\item Superviser les stagiaires.
			\end{itemize}

			\subsubsection*{Janvier 2017 – Septembre 2017 \\ Nikol Poulin – Commerce en ligne}
			\begin{itemize}
				\scriptsize
				\item Mettre en place un espace représentant et un espace client.
				\item Faire la refonte incluant la création du design et de l'intégration.
				\item Créer un module permettant aux clients de soumissionner.
				\item Faire le changement de technologie ASP vers le PHP.
				\item Coordoner une API qui transforme une BD Acomba en MySQL.
				\item Automatisation de tâches opérant à la création de rapports.
			\end{itemize}

			\subsubsection*{Février 2016 – Août 2016 \\ Trillium Construction – Gestion de la relation client}
			\begin{itemize}
				\scriptsize
				\item Analyser les besoins.
				\item Développer l'application Web sous l'approche mobile first.
				\item Mettre en place un système de gestion d'inventaire et de projets.
				\item Concevoir un environnement de gestion comptable.
			\end{itemize}

		\columnbreak

		\subsection*{Expériences professionels}
			\subsubsection*{Juin 2014 – Présent \\ Travailleur autonome – Développeur Web Full-Stack}
			\begin{itemize}
				\scriptsize
				\item Rédiger des documents d’analyse et des soumissions.
				\item Rencontrer de nouveaux clients afin de leur proposer des solutions.
				\item Développer des E-commerces, des logiciels de gestion et des CRM.
				\item Programmation PHP et javascript des logiciels.
				\item Maintenir et améliorer de systèmes.
			\end{itemize}

			\subsubsection*{Juin 2014 – Présent \\ Logiaction – Développeur Web Full-Stack \& Pigiste}
			\begin{itemize}
				\scriptsize
				\item Réaliser les demandes des clients sur leur site web.
				\item Rehausser la sécurité de systèmes déjà existants.
				\item Transformer de vieux sites web ASP en PHP.
				\item Développer des systèmes de gestion personalisés.
			\end{itemize}
			\link{https://www.logiaction.com}{www.logiaction.com}

			\subsubsection*{Mars 2014 – Juin 2014 \\ Cégep de Sainte-Foy – Développeur Web \& Stagiaire}
			\begin{itemize}
				\scriptsize
				\item Concevoir des maquettes de systèmes de gestion de contenu.
				\item Intégrer et programmer des systèmes informatiques.
				\item Modifier des modules existants afin de les rendre plus sécuritaires.
			\end{itemize}
			\link{http://www.cegep-ste-foy.qc.ca}{www.cegep-ste-foy.qc.ca}
	\end{multicols}

	\vspace{2.5mm}

	\begin{multicols}{2}
		\subsection*{Projets libres}
			\subsubsection*{Build Your Own Arch Linux Repository}
			\paragraph{Ce projet est un programme facile à utiliser qui permet de créer et maintenir votre propre répertoire Arch Linux.}
			\vspace{1mm}
			\link{https://github.com/unix-development/build-ar}{www.github.com/unix-development/build-ar}

			\subsubsection*{Form Async}
			\paragraph{Form Async est une librarie JavaScript qui permet d'envoyer des requètes Ajax lorsqu'un champs de formulaire change.}
			\vspace{1mm}
			\link{https://github.com/lognoz/form-async}{www.github.com/lognoz/form-async}

			\subsubsection*{Embla}
			\paragraph{Embla est une configuration pour Emacs écrit en Lisp avec une multitude de raccourcis clavier.}
			\vspace{1mm}
			\link{https://github.com/lognoz/embla}{www.github.com/lognoz/embla}

		\columnbreak

		\subsection*{Expertises}

		{\setlength\multicolsep{0pt}
		 \begin{multicols}{2}
			\subsubsection*{Programmation Front-End}
			\begin{itemize}
				\scriptsize
				\item Compass, SASS, LESS
				\item JQuery, JavaScript
				\item HTML5
			\end{itemize}

			\subsubsection*{Gestion \& Analyse}
			\begin{itemize}
				\scriptsize
				\item Org mode
				\item Axure
				\item LaTeX
			\end{itemize}

			\subsubsection*{Système d'exploitation}
			\begin{itemize}
				\scriptsize
				\item Linux
				\item OpenBSD
				\item Windows
				\item MacOS
			\end{itemize}

			\columnbreak

			\subsubsection*{Programmation Back-End}
			\begin{itemize}
				\scriptsize
				\item PHP
				\item Python
				\item Lisp
				\item C
				\item Makefile
				\item Bash
				\item MySQL
			\end{itemize}

			\subsubsection*{Traitement des médias}
			\begin{itemize}
				\scriptsize
				\item Adobe Photoshop
				\item Adobe Illustrator
				\item Adobe Lightroom
				\item DaVinci Resolve Studio
			\end{itemize}
		 \end{multicols}}
	\end{multicols}

	\vspace{2.5mm}

	\begin{multicols}{2}
		\subsection*{Éducation}
			\subsubsection*{2011 – 2014 \\ Cégep de Sainte-Foy – Technique d'intégration multimédia}
			\paragraph{Cours de spécialisation réalisé sur Symfony2.}

		\columnbreak

		\subsection*{J'aime}
			{\setlength\multicolsep{0pt}
			 \begin{multicols}{2}
				\subsubsection*{Cinéma}
				\begin{itemize}
					\scriptsize
					\item Réaliser des courts-métrages
					\item Regarder des films
				\end{itemize}

				\subsubsection*{Informatique}
				\begin{itemize}
					\scriptsize
					\item Sécurité et anonymat
					\item Linux et OpenBSD
				\end{itemize}
			 \end{multicols}}
	\end{multicols}
\end{document}
